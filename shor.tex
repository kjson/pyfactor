	As of now, the GNFS is the best currently know algorithm for factoring integers. However, it is still an exponential time algorithm. We can see that our current method of computations with bits isn’t working very well. So the idea of Shor’s algorithm is to use quantum computations using quantum bits, also known as qubits. Shor's Algorithm was developed by Peter Shor in 1994. \\

Shor’s algorithm has three components: phase estimation, order finding and factoring. The algorithm is based on the following: phase estimation $\Rightarrow$ order finding $\Rightarrow$ factoring. In other words, if you can do phase estimation, you can do order finding and hence can do factoring. \\


\subsection{Order Finding}
	The order finding problem is defined as follows: We are given a positive integer $N > 2$ and another positive integer $x$ with $x < N$ and $x$ co-prime to $N$ (they share no common factors). The order finding problem is to find the smallest positive integer $r$ such that $x^r$ mod $N = 1$. We say that $r$ is the order of $x$ in $N$. Currently, this problem is hard to do using the normal computing power that everyday computers use. So, we must use quantum computing with phase estimation to try and solve this problem.

\subsection{Phase Estimation}
	The phase estimation problem is defined as follows: We are given a quantum circuit $Q$ that performs a unitary operation $U$ and given a quantum state $\ket{\psi}$ that is promised to be an eigenvector of $U$, i.e. $U \ket{\psi} = e^{2πiθ} \ket{\psi}$. The phase estimation problem is to find an approximation of $\theta$ in $[0, 1)$. Note that no specific requirements have been placed on the accuracy to which $θ$ must be approximated. This problem is solved by using entanglement with quantum circuits and the quantum Fourier transform. Therefore, we can now do phase estimation and hence order finding and factoring.

\subsection{Algorithm}

	Shor’s Algorithm is described as follows: We have a positive odd integer $N$ that is not a prime power that we want to factor.

	\begin{enumerate}
		\item Pick a random integer a such that $2<= a < N$.
		\item Find $\gcd(a, N)$ using the Euclidean algorithm.
		\item If the $\gcd(a, N)$ is not $1$, then the $\gcd(a, N)$ is a nontrivial factor of $N$ and stop.
		\item Otherwise, we find the order $r$ of the element a using the order finding algorithm (i.e. using phase estimation to do order finding)
		\item If $r$ is odd, stop.
		\item Otherwise, find $x = a^(r/2) - 1$ and then the $\gcd(x, N)$
		\item If $\gcd(x, N)$ is not $1$, then $d$ is a nontrivial factor of $N$ and stop.
	\end{enumerate}

	Keep repeating steps 1 – 7 with different a values until you have factored N as a product of prime powers, or until you are satisfied with the algorithm’s current result.

\subsection{Complexity}

	If $N$ is the number to be factored, Shor’s algorithm takes $O(log(N)^3)$ time. More precisely, it is $O((\log N)^2(\log \log N)(\log \log \log N))$ when you are using fast exponentiation to compute the powers of the numbers. So the algorithm is in polynomial time. At this point in time, you may be thinking that since we now have a polynomial time algorithm for the Integer Factorization problem, we can now say the $FP = FNP$ and hence $P = NP$. Unfortunately, Shor’s algorithm only proves that the problem is in the complexity class $BQP$, which is the class of problems solvable by a quantum computer in polynomial time. Currently, we know that $P$ is a subset of $BQP$, but we do not know whether or not $BQP$ is a subset of $P$. If it was, then we could then come to the final conclusion that $P = NP$. 

\subsection{Downside}
	Another problem we currently face with Shor’s algorithm is that we currently do not have an efficient quantum computer, so this algorithm is still impossible to properly run. Currently, the largest integer that has been factored by quantum computers is $56153 = 233 \cdot 241$. This was done by Nike Dattani at Kyoto University and Oxford University, along with Nathaniel Bryans at Microsoft (who is now at the University of Calgary) in November $2014$ Previously, the largest integer that had been factored by a quantum computer using Shor’s algorithm was $143 = 11 \cdot 13$.This was done by physicists at the University of Science and Technology of China in Hefei, China in 2012. 
